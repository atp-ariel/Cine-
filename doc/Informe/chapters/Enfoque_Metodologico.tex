\chapter{Enfoque Metodológico}

Para el desarrollo de este proyecto aplicaremos la metodología ágil Extreme Programming(XP).

El listado de los requisitos del proyecto en cuestión se encuentra bien detallado, pero debido a la inexperiencia del personal, el desarrollo de este puede sufrir cambios por lo que el empleo de esta metodología facilitará la asimilación de estos.

Contamos con la disponibilidad del cliente para todos los encuentros en los que sea necesario debatir y valorar las sugerencias que brinde nuestro equipo y el cliente para lograr simplicidad, lo cual propiciará retroalimentación frecuente entre ambas partes, así como una mayor oportunidad de dirigir el esfuerzo eficientemente, lo cual es una práctica de esta metodología (Cliente in-situ).

Se realizarán entregas sobre los avances en el software cada dos semanas.

El personal tiene como principio hacer un software de calidad, en el que se apliquen los principios SOLID, DRY, KISS, YAGNI, una arquitectura que permita el desacoplamiento y extensibilidad, además que constituye una exigencia del cliente. De esta manera pondríamos en práctica la refactorización y el diseño simple.

Como característica del personal se permite que cualquier programador puede cambiar cualquier parte del código en cualquier momento motivando a la aparición de nuevas ideas por parte del personal.

Cada uno de los motivos anteriores expuestos conllevó a la utilización de esta metodología en el proyecto asignado.